\documentclass[12pt,a4paper]{article}
\usepackage[utf8]{inputenc}
\usepackage[czech]{babel}
\usepackage[T1]{fontenc}
\usepackage{amsmath}
\usepackage{amsfonts}
\usepackage{amssymb}
\usepackage{graphicx}
\usepackage{epstopdf}
\usepackage{indentfirst}
\setlength{\parindent}{4em} 
\author{Jakub Drápela}
\usepackage{fancyhdr}
\usepackage{siunitx}
\usepackage{pdflscape}


\graphicspath{{./imgs/}}

%\fontfamily{phs}
%\selectfont

\begin{document}
\pagestyle{empty}

%%nastaveni pisma  
%\fontfamily{phv}
%\selectfont

	\begin{center}

\large

České vysoké učení technické v Praze

\medskip

Fakulta elektrotechnická 
\vfill
\vfill
{\LARGE\bfseries Household Intelligent Assistant}


\vspace{9mm}

\begin{figure}[h!]
\begin{center}
\includegraphics[width = 10cm]{ucho.jpg} 
\end{center}
\end{figure}

\vspace{9mm}

\begin{tabular}{rl}

Autoři: & Jiří Burant \\
\noalign{\vspace{1mm}}
		& Jakub Drápela \\
		\noalign{\vspace{1mm}}
		& Martin Klučka\\
		\noalign{\vspace{1mm}}
		& Petr Kovář \\
		\noalign{\vspace{1mm}}
		& Jakub Konrád\\
		\noalign{\vspace{1mm}}
		& Pavel Trutman\\
\noalign{\vspace{2mm}}
Studijní obor: & Kybernetika a robotika \\
\noalign{\vspace{2mm}}
Datum vypracování: & \today\\
\end{tabular}

\end{center}

\newpage
\pagestyle{plain}     % zapne obyčejné číslování
\setcounter{page}{1}
%% zahlaví a zápatí
\addtolength{\voffset}{-3cm}
\addtolength{\headheight}{2cm}

\pagestyle{fancy}
\lhead{\includegraphics[scale=0.12]{cvut_text.jpg}  }
\rhead{\textbf{Household Intelligent Assistant}}
%\rhead{\textit{\bfseries Burant,Drápela,Klučka,Kovář,Konrád,Trutman}}
\lfoot{}
\cfoot{\thepage}
\rfoot{}
\renewcommand{\headrulewidth}{0.4pt}


\section*{Popis projektu, motivace}
Jako lidi si stále klademe různé otázky. Abychom mohli ve světě rozumně fungovat potřebujeme na tyto otázky znát odpovědi. Od té doby, co se začaly informace zaznamenávat lze odpovědi nalézt v záznamech. V nedávné historii lidé prahnoucí po informacích ve velkém kupovali knihy, noviny, jízdní řády, prostě média se žádaným obsahem. 

Doba pokročila a my jsme se posunuli do epochy, ve které jsou téměř všechny informace uložené v elektronické podobě. Je běžné k nim přistupovat pomocí moderních přístrojů. Téměř každý aktivní člověk vlastní alespoň jednu z věcí, jako je počítač, notebook, chytrý mobilní telefon, tablet, iPad a jiné. S využitím toho vybavení máme značně usnadněnou cestu k získání potřebné informace. Stačí mít u sebe správnou aplikaci, přístup na internet a například dopravní spojení nalezneme do minutky. V dnešní době nás limituje už jen to, že si požadovanou informaci musíme najít sami.

My se snažíme udělat krok vpřed. Ruční hledání informací obejít a nechat si informaci vyhledat automaticky nástrojem, který můžeme ovládat třeba jednoduše hlasem.

\section*{Detailní specifikace}
V našem projektu se zaměříme na vytvoření dialogového systému pro použití v běžné domácnosti či kanceláři. Tedy systému, který dovede sám odpovídat na kladené otázky z limitované oblasti počasí, dopravy, obecných informací atd. Systém bude neustále čekat na aktivační slovo a potom bude schopen zodpovědět určitou škálu otázek. K realizaci projektu použijeme již existující aplikace, které vhodně propojíme do funkčního celku.

V současné době existuje řada aplikací, které fungují na podobném principu. Uveďme například aplikaci \textbf{Cortana} od firmy Microsoft. Cortana, podobně jako další aplikace \textbf{Siri} od firmy Apple, je označována jako osobní asistent. Lze ji použít pro širokou škálu úkonů. Například hlasové ovládání přijímače nebo vyhledávání informací na internetu. Její nevýhodou je použití pouze pod operačním systémem Windows, případně iOS. 

Jako další lze uvést open-source aplikaci \textbf{Jasper} v jazyce Python. Tato aplikace umožňuje využití hlasu pro získání informací. Můžeme ji také použít v inteligentním bydlení a v dalších situacích. Jelikož je volně dostupná, lze k ní libovolně přidávat nové moduly a její použití ještě rozšířit. Stává se tak pro nás vhodnou inspirací. 

Za zmínku ještě stojí projekt \textbf{Amazon echo} od firmy Amazon. Tento osobní asistent z titulní strany umí přehrávat hudbu, odpovídat na otázky, vyhledávat zprávy, zjišťovat počasí, vytvářet seznamy a mnoho dalších věcí. 

My se budeme snažit o sestavení vlastního systému s podobnými vlastnostmi, jako mají již vzniklé projekty. 

\section*{Popis navrženého řešení}
Samotná finální aplikace se bude skládat z aplikačního jádra a modulů, které budou zodpovědné za jednotlivé funkcionality aplikace. Jádro aplikace bude pouze volat jednotlivé moduly a bude jim jen předávat získaná data z ostatních modulů. Bude tedy pouze řídit celý běh aplikace, ale nebude v něm probíhat žádné zpracování informací. Naopak jednotlivé moduly budou úzce zaměřené na zpracování nebo zjištění konkrétních dat. Mezi jádrem aplikace a jednotlivými moduly bude striktně definované rozhraní, což nám snadno umožní jednotlivé moduly nahradit za jiné, aniž by bylo třeba upravit jádro aplikace nebo ostatní moduly. Protože je aplikace vyvíjena jako otevřený software, umožňuje tato vysoká modularita každému uživateli jednoduše vytvořit si vlastní modul, případně si stávající modul upravit. Blokové schéma aplikace je zobrazeno na obrázku \ref{fig:diagram api}.

\begin{figure}[ht]
	\begin{center}
	\includegraphics[height = 9cm]{blockDiagram.pdf}
	\caption{Blokový diagram navrženého řešení projektu Household Intelligent Assistant.}
	\label{fig:diagram api}
	\end{center}
\end{figure}

Při popisu návrhu řešení aplikace budeme postupovat tak, jak bude reakce na mluvené slovo postupně prostupovat mezi jednotlivými moduly, až nakonec vznikne finální řečená odpověď.

Za běžného provozu bude aplikace ve stand-by režimu, ve kterém bude čekat na zaznění aktivačního slova. Na rozpoznání aktivačního slova bude vyhrazen jeden samostatný modul. Aktivační slovo by mělo být snadno rozlišitelné od ostatních a zvukově velmi výrazné, aby nedocházelo k jeho záměně s hlukem na pozadí. Ukazuje se vhodné zvolit tříslabičné slovo s výraznými písmeny, jako jsou například písmena \uv{r} nebo \uv{x}.

Po zaznění aktivačního slova uživatelem se aktivuje blok pro rozpoznání mluveného slova, který převede položenou otázku na prostý text. Vhodnou robustní aplikací pro tento problém je třeba vybrat. Jako vhodné aplikace se ukazují třeba \textit{wit.ai} nebo \textit{PocketSphinx}. Tyto aplikace podrobněji prozkoumáme a vybereme tu, která se bude na daný problém lépe hodit.

Ze získaného textu z předchozího kroku je třeba vybrat klíčová slova, podle kterých určíme význam otázky. Klíčová slova umí rozeznávat, například již zmiňovaná, aplikace \textit{PocketSphinx}.

Získaná klíčová slova jsou vstupní hodnotou do modulu, jenž získá potřebné informace z internetu. Jednotlivé okruhy informací budou obstarávat jednotlivé subsystémy. Jeden subsystém bude například pouze na zpracování odpovědi na počasí, třeba přes aplikaci \textit{forecast.io}.

Informace získané z internetu je třeba přeformulovat do srozumitelné odpovědi podle předem zadaných vzorů. To bude úkolem dalšího samostatného modulu.

V poslední řadě využijeme aplikaci k převedení formulované odpovědi do strojově mluvené řeči. Prvotní aplikací pro tento převod může být jednoduchý \textit{"The Festival Speech Synthesis System"}. Později můžeme použít jinou aplikaci, která má převod textu na řeč více propracovanější.

% plus dopsat nějaké kecy kolem 

\section*{Plánování projektu (Ganttův diagram, úkoly, milníky)}
\begin{landscape}
~\vfill
\begin{figure}[ht]
	\begin{center}
	\includegraphics[height = 0.6\textheight ]{PTO-Gantt.png}
	\caption{Ganttův diagram s milníky a úkoly.}
	\label{fig:diagram gantt}
	\end{center}
\end{figure}
\vfill
\end{landscape}

\section*{Analýza rizik a krizové plány}
V průběhu řešení projektu mohou nastat situace, jenž budou nepříjemnou překážkou v realizaci projektu. V nejhorším případě zapříčiní jeho zkázu. S riziky je třeba počítat a připravit si pro mě krizové plány. Lze se tak vyvarovat zbytečných zmatků a vzniklý problém efektivně řešit. \\

 Rizika projektu, která mohou nastat a plán na jejich eliminaci: 

\begin{itemize}
	\item{\textbf{Nedostatečná funkčnost některého z modulů}} - jednotlivé moduly, které využívají již existující aplikace nebudou mít dostatečnou funkčnost.
	
	\item{\textbf{Nemoc}} - v dlouhodobém řešení projektu zastihne některého člena týmu vážná nemoc. 
	
	\item{\textbf{Finance}} - nebude k dispozici dostatek financí.
	
	\item{\textbf{Ztráta dat}} - vlivem selhání lidského faktoru nebo elektroniky dojde ke ztrátě kódu a příslušných souborů.
	
\end{itemize}

\section*{Předběžné výsledky}

Zdroje:
http://static01.nyt.com/images/2015/06/25/business/GADGETWISE/GADGETWISE-master675.jpg
\end{document}
